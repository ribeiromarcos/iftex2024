\documentclass[artigo]{iftex2024}

\addbibresource{referencias.bib}
\titulo{Título do projeto de trabalho de conclusão de curso}
\autor{Marcos Roberto Ribeiro}
\orientador{Nome do Orientador}
\coorientador[F]{Nome da coorientadora}
\email{marcos@ifmg.edu.br}
\submissao{2024-02-15}
\data{2024-04-30}
\curso{Bacharelado}{Engenharia de Computação}

\begin{document}

\maketitle

\section{APRESENTAÇÃO DO TEMA PROPOSTO}

Neste espaço deverá ser feita uma apresentação sucinta do objeto de estudo do futuro projeto, dentro de uma abordagem que torne sua investigação viável em um projeto de pesquisa ou TCC.

É importante observar o regulamento e as normas de formatação e de elaboração de trabalhos de conclusão de curso do IFMG \cite{ifmg:2020:manual,ifmg:2021:tcc}.
Além disso, é interessante consultar o manual da classe {\iftex} para conhecer mais sobre as configurações e exemplos de uso \cite{ribeiro:2024:iftex}.

\section{REFERENCIAL TEÓRICO}

Neste espaço deverá apresentar a revisão bibliográfica realizada para sustentar e propor o seu objeto de estudo. Devem ser definidos os principais conceitos e elencados os trabalhos correlatos ao tema de pesquisa. Uma maior ênfase deve ser dada ao estado-da-arte.

\section{OBJETIVOS}

\subsection{Objetivo Geral}

Neste espaço deverá ser redigido, com verbo no infinitivo, um pequeno texto sintetizando o que se deseja alcançar ao final do trabalho. Este objetivo está associado a uma visão mais global do tema do trabalho.

\subsection{Objetivos Específicos}

Neste espaço deverá ser redigido, com verbo no infinitivo, um texto explicitando mais detalhadamente os objetivos finais do trabalho. Estes objetivos são desdobramentos dos objetivos anteriores e estão associados a uma visão mais tecnicista do trabalho e podem ser vistos como passos para o alcance dos objetivos gerais.

\subsection{Resultados Esperados}

Neste espaço apresente quais são os resultados que se pretende alcançar com o desenvolvimento do trabalho, bem como suas possíveis contribuições científicas, tecnológicas, sociais, econômicas e ambientais.

\section{JUSTIFICATIVA}

Neste espaço deve ser redigido um pequeno texto explicitando os motivos que justificam a elaboração de um trabalho acadêmico sobre o tema proposto.

\section{METODOLOGIA}

Neste espaço devem ser esclarecidas as técnicas e/ou métodos utilizados para a investigação do tema proposto. O texto deve responder à seguinte questão: COMO realizar o trabalho a respeito do tema proposto?

\section{CRONOGRAMA}

O Quadro \ref{qua:cronograma} exibe o cronograma de execução do projeto.

\begin{board}[!htb]
\caption{Cronograma de execução do projeto} \label{qua:cronograma}
\begin{tabularx}{\textwidth}{|X|c|c|c|c|c|c|} \hline
\bfseries Atividades & \bfseries Mês 1 & \bfseries Mês 2 & \bfseries Mês 3 & \bfseries Mês 4 & \bfseries Mês 5 & \bfseries Mês 6 \\ \hline
Atividade 1            &    X    &         &         &         &         &         \\ \hline
Atividade 2            &         &    X    &         &         &         &         \\ \hline
Atividade 3            &         &         &    X    &         &         &         \\ \hline
Atividade 4            &         &         &         &    X    &         &         \\ \hline
Atividade 5            &         &         &         &         &    X    &         \\ \hline
Atividade 6            &         &         &         &         &         &    X    \\ \hline
\end{tabularx}
\legend{Elaborado pelo autor, 2023.}
\end{board}

\section{ORÇAMENTO}

Se for o caso! Descrevendo, inclusive, a origem do dinheiro/recurso a ser utilizado.

\section*{REFERÊNCIAS}

\printbibliography

\end{document}
