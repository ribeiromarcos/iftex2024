\documentclass[estagio]{iftex2024}

\addbibresource{referencias.bib}

\titulo{Modelo de Relatório de Estágio}
\autor{Marcos Roberto Ribeiro}
\local{Bambuí - MG}
\data{2024-01-26}

\campus{\textit{Campus} Bambuí}
\curso{Bacharelado}{Engenharia de Computação}
\titulacao{Bacharel}

\orientador{Nome do Orientador}
\empresa{Capsule Corporation}
\horas{240}

\membrobanca{Fulando de Tal}{Instituição do Fulano de Tal}
\membrobanca{Ciclano de Tal}{Instituição do Ciclano de Tal}

\assinaturas{assinaturas.pdf}

\resumo{Este trabalho é um modelo em {\latex} utilizando a classe \iftex.
Tal classe foi desenvolvida com base no manual de normalização de trabalhos acadêmicos do IFMG e nas normas relacionadas da Associação Brasileira de Normas Técnicas.
Este modelo apresenta uma estrutura básica com exemplos de elementos pré e pós-textuais.
Maiores informações sobre como utilizar a classe podem ser encontradas no manual da classe \iftex.}
\palavraschave{\iftex. Modelo. IFMG. \latex.}

\listafiguras
\listaquadros
\listatabelas

\begin{document}

\maketitle

\chapter{INTRODUÇÃO}

A introdução desempenha um papel fundamental na preparação do leitor para o conteúdo que será abordado.
Ela começa contextualizando o tema, fornecendo informações relevantes sobre o assunto, sua importância e seu contexto mais amplo na área de estudo.
Além disso, a introdução deve fornecer justificativas convincentes para a realização da pesquisa, identificando lacunas no conhecimento existente, relevância prática ou teórica do tema e importância potencial dos resultados.

Destaca-se que as contribuições esperadas do trabalho para a área de estudo, que podem incluir avanços teóricos e práticos, implicações políticas ou sociais, entre outros.
Por fim, a introdução é geralmente concluída com um parágrafo que resume brevemente o objetivo geral do trabalho, reiterando os objetivos estabelecidos anteriormente.
É essencial que essa seção seja redigida com clareza e coesão para capturar a atenção do leitor e estabelecer uma base sólida para o restante do trabalho.

Por fim, é importante observar o regulamento e as normas de formatação e de elaboração de trabalhos de conclusão de curso do IFMG \cite{ifmg:2020:manual,ifmg:2021:tcc}.
Além disso, é interessante consultar o manual da classe \iftex para conhecer mais sobre as configurações e exemplos de uso \cite{ribeiro:2024:iftex}.

\section{Objetivos}

Os objetivos definem claramente o propósito e as metas do trabalho, devem ser específicos, mensuráveis, alcançáveis, relevantes e limitados no tempo.
Os objetivos podem ser divididos em objetivo geral e objetivos específicos.

\chapter{CARACTERIZAÇÃO DO ESTÁGIO}

Este capítulo informações da empresa, áreas de atuação no estágio e do supervisor.

\section{Identificação do campo de estágio}

Nesta seção, são apresentadas informações específicas sobre as entidades envolvidas no estágio, tais como o supervisor, o local onde o estágio foi realizado e a empresa.

\subsection{Identificação do local do concedente do estágio}

O Quadro \ref{boa:empresa} mostra a identificação da empresa concedente do estágio.

\begin{board}[!htb] \centering
\caption{Identificação da empresa} \label{boa:empresa}
\begin{varwidth}{\linewidth}
  \begin{tabular}{|r|l|} \hline
  Nome:           & Nome da Empresa                  \\ \hline
  Endereço:       & Rua Santos Dumon                 \\ \hline
  Cidade:         & Belo Horizone -- MG              \\ \hline
  Telefone:       & (31) 0000-0000                   \\ \hline
  Site:           & \url{http://www.empresa.com.br}  \\ \hline
  E-mail:         & contato@empresa.com.br           \\ \hline
  \end{tabular}
  \legend{Elaborado pelo autor, 2024.}
\end{varwidth}
\end{board}

\subsection{Áreas onde foi realizado o estágio}

O estágio foi realizado na área de desenvolvimento de sistemas, automatização de processos e extensões de plataformas na linguagem JavaScript.
O Qaudro \ref{boa:periodo} apresenta o período de realização do estágio.

\begin{board}[!htb] \centering
\caption{Período de estágio} \label{boa:periodo}
\begin{varwidth}{\linewidth}
  \begin{tabular}{|r|l|} \hline
  Data de início:        & 17/01/2024 \\ \hline
  Data de término:       & 09/03/2024 \\ \hline
  Carga horária semanal: & 30 horas   \\ \hline
  Carga horária total:   & 222 horas  \\ \hline
  \end{tabular}
  \legend{Elaborado pelo autor, 2024.}
\end{varwidth}
\end{board}


\subsection{Supervisor do Estágio}

O Quadro \ref{boa:supervisor} exibe as informações do supervisor de estágio.

\begin{board}[!htb] \centering
\caption{Supervisor de estágio} \label{boa:supervisor}
\begin{varwidth}{\linewidth}
  \begin{tabularx}{\linewidth}{|r|X|} \hline
  Nome:             & Nome do Supervisor de Estágio \\ \hline
  Formação:         & Engenharia de Computação pelo Instituto Federam de Minas Gerais (IFMG) em 2017, Doutorado em Ciência da Computação pela Universidade Federal de Minas Gerais (IFMG) 2023. \\ \hline
  Cargo na empresa: & Diretor de Tecnologia   \\ \hline
  \end{tabularx}
  \legend{Elaborado pelo autor, 2024.}
\end{varwidth}
\end{board}


\section{Apresentação do local e da Empresa} \label{sec:apresentacao}

Esta seção apresenta mais informações sobre a empresa na qual o estágio foi realizado.

\subsection{A empresa} \label{subsec:empresa}

A empresa {\theEmpresa} foi fundada em 2020 com o propósito de criar soluções customizadas para cada cliente.
A equipe de desenvolviemnto da empresa tem consideravel experiência nas áreas de Desenvolvimento para Internet e Inteligência Artificial.
Essencialmente, a empresa promove a aprendizagem adicional dos colaboradores, além das competências específicas de sua área, visando aprimorar a experiência e a comunicação.
Atualmente, a empresa conta com cerca de 50 colaboradores, sendo que sua sede está localizada em Medeiros -- MG.

\subsection{Missão, visão e valores}

A {\theEmpresa} tem como missão maximizar os resultados de seus clientes, oferecendo soluções personalizadas e inovadoras.
Com uma equipe experiente e multidisciplinar, a empresa busca liderar o mercado, entregando excelência e inovação.
Seus valores fundamentais são a excelência, inovação, colaboração, ética, conhecimento e impacto.

\subsection{Local de trabalho}

O estágio foi realizado de forma remota, no Departamento de Tecnologia da empresa.
A plataforma Discord\footnote{\url{https://discord.com/}} foi utilizada como meio de comunicação entre os membros das equipes durante os projetos desenvolvidos.

\chapter{ATIVIDADES DESENVOLVIDAS}

As atividades desenvolvidas envolvem o detalhamento das atividades realizadas durante o estágio.
No desenvolvimento, podem ser utilizadas figuras, tabelas e quadros para ilustrar melhor a evolução do trabalho.
A Figura \ref{figura:logomarca_if} exibe a logomarca dos institutos federais.
Outros exemplos são a Tabela \ref{tabela:lista_produtos} e o Quadro \ref{quadro:editores_texto_livres}.

\begin{figure}[!htb] \centering
  \caption{Logomarca do IF} \label{figura:logomarca_if}
  \begin{varwidth}{\linewidth}
    \includegraphics[width=4cm]{figuras/if}
    \legend{\citefonte{ifmg:2020:manual}.}
  \end{varwidth}
\end{figure}

\begin{table}[!htb]
\caption{Lista de produtos} \label{tabela:lista_produtos}
\begin{tabularx}{\textwidth}{X|l|r|r|r} \hline
Produto      & Unidade & Preço (R\$) & Quantidade & Total (R\$) \\ \hline
Arroz        & Kg      & 2,00        & 550        & 1.100,00    \\
Óleo de Soja & L       & 2,50        & 500        & 750,00      \\
Açucar       & Kg      & 3,00        & 100        & 300,00      \\ \hline
\end{tabularx}
\legend{Elaborado pelo Autor, 2020.}
\end{table}

\begin{board}[!htb] \centering
\caption{Editores de Texto Livres} \label{quadro:editores_texto_livres}
\begin{varwidth}{\linewidth}
\begin{tabular}{|l|l|r|}        \hline
Editor     & Multiplataforma & Específico para Latex \\ \hline
Kwriter    & Sim             & Não                   \\
Texmaker   & Sim             & Sim                   \\
Kile       & Sim             & Sim                   \\
Geany      & Sim             & Não                   \\ \hline
\end{tabular}
\legend{Elaborado pelo Autor, 2020.}
\end{varwidth}
\end{board}

\chapter{CONCLUSÃO}

A conclusão resume os principais pontos discutidos e apresenta as conclusões alcançadas a partir do trabalho.
Ela destaca as descobertas mais significativas, sua relação com a literatura existente e suas implicações práticas ou teóricas.
Além disso, a conclusão reafirma os objetivos do trabalho e sugere áreas para futuras investigações.
É importante evitar a introdução de novas informações e manter a conclusão concisa e alinhada com os objetivos e resultados do estudo.

Após a conclusão são apresentados alguns exemplos de elementos pós-textuais.
Inclusive, elementos como apêndices e anexos devem ser referenciados.
Como exemplo, exitem o Apêndice \ref{ap:exemplo} e o Anexo \ref{an:exemplo}.

\chapter*{REFERÊNCIAS}

\printbibliography

\appendix

\chapter{Exemplo de apêndice} \label{ap:exemplo}

Este é apenas um exemplo de apêndice.

\attachment

\chapter{Exemplo de anexo} \label{an:exemplo}

Este é apenas um exemplo de anexo.

\end{document}
