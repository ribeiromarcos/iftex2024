\documentclass[atividade]{iftex2024}

\addbibresource{referencias.bib}
\titulo{Modelo de Atividade}
\autor{Marcos Roberto Ribeiro}
\local{Bambuí -- MG}
\data{2024-04-22}
\instituicao[IFMG]{Instituto Federal de Minas Gerais}
\campus{\textit{Campus} Bambuí}
\curso{Bacharelado}{Engenharia de Computação}
\professor[F]{Nome da Professora}
\disciplina{Banco de Dados}

\begin{document}

\maketitle

\chapter{INTRODUÇÃO}

A introdução desempenha um papel fundamental na preparação do leitor para o conteúdo que será abordado.
Ela começa contextualizando o tema, fornecendo informações relevantes sobre o assunto, sua importância e seu contexto mais amplo na área de estudo.
Além disso, a introdução deve fornecer justificativas convincentes para a realização da pesquisa, identificando lacunas no conhecimento existente, relevância prática ou teórica do tema e importância potencial dos resultados.

Destaca-se que as contribuições esperadas do trabalho para a área de estudo, que podem incluir avanços teóricos e práticos, implicações políticas ou sociais, entre outros.
Por fim, a introdução é geralmente concluída com um parágrafo que resume brevemente o objetivo geral do trabalho, reiterando os objetivos estabelecidos anteriormente.
É essencial que essa seção seja redigida com clareza e coesão para capturar a atenção do leitor e estabelecer uma base sólida para o restante do trabalho.

Por fim, é importante observar o regulamento e as normas de formatação e de elaboração de trabalhos de conclusão de curso do IFMG \cite{ifmg:2020:manual,ifmg:2021:tcc}.
Além disso, é interessante consultar o manual da classe \iftex para conhecer mais sobre as configurações e exemplos de uso \cite{ribeiro:2024:iftex}.

\chapter{FUNDAMENTOS TEÓRICOS}

A seção de fundamentos teóricos fornece uma base teórica sólida para o estudo, contextualizando o trabalho dentro do corpo existente de conhecimento na área.
A seção de fundamentos teóricos fornece a base conceitual e contextual para o seu estudo.
É importante escrevê-la de forma clara, organizada e fundamentada em pesquisas anteriores, destacando a relevância e originalidade do seu trabalho.

\chapter{METODOLOGIA}

A metodologia descreve os métodos e procedimentos utilizados na pesquisa.
Ela inclui detalhes sobre o design do estudo, a coleta e análise de dados, além da justificativa das escolhas metodológicas.
É essencial para garantir a validade e confiabilidade dos resultados.
A metodologia deve ser clara e detalhada o suficiente para que outros pesquisadores possam replicar o estudo.

De acordo com \citet{ifmg:2021:tcc}, \enquote{Todos os trabalhos devem informar, no capítulo referente à Metodologia, a sua classificação quanto à natureza, objetivo, procedimentos e abordagem}.
Uma possível classificação pode ser feita quanto à abordagem.
Nesse caso, uma exemplo de classificação é a pesquisa qualitativa:
\begin{quote}
A pesquisa qualitativa preocupa-se com aspectos da realidade que não podem ser quantificados, centrando-se na compreensão e explicação da dinâmica das relações sociais.
Não se preocupa com representatividade numérica em si, mas com o aprofundamento da compreensão de um grupo social, de uma organização, etc.
\cite{ifmg:2021:tcc}.
\end{quote}

\chapter{DESENVOLVIMENTO}

A seção de desenvolvimento apresenta e discute os resultados do trabalho.
Inicialmente, os resultados são apresentados de forma objetiva, seguidos por uma discussão que os relaciona aos objetivos e à revisão de literatura.
A interpretação dos resultados à luz das teorias é essencial, assim como a comparação com estudos anteriores.
Finalmente, é importante reconhecer as limitações do estudo e sugerir direções futuras.
Essa seção contribui para a compreensão do tema e o avanço do conhecimento na área.

\chapter{CONCLUSÃO}

A conclusão resume os principais pontos discutidos e apresenta as conclusões alcançadas a partir do trabalho.
Ela destaca as descobertas mais significativas, sua relação com a literatura existente e suas implicações práticas ou teóricas.
Além disso, a conclusão reafirma os objetivos do trabalho e sugere áreas para futuras investigações.
É importante evitar a introdução de novas informações e manter a conclusão concisa e alinhada com os objetivos e resultados do estudo.

\chapter*{REFERÊNCIAS}

\printbibliography

\end{document}
