\documentclass{iftex2024}

\addbibresource{referencias.bib}

\titulo{Modelo de Relatório Técnico}
\autor{Marcos Roberto Ribeiro}
\local{Bambuí - MG}
\data{2024-01-26}
\campus{\textit{Campus} Bambuí}
\curso{Bacharelado}{Engenharia de Computação}
\titulacao{Bacharel}

\orientador{Nome do Orientador}
\membrobanca{Fulando de Tal}{Instituição do Fulano de Tal}
\membrobanca{Ciclano de Tal}{Instituição do Ciclano de Tal}

\fichacatalografica{ficha.pdf}
\assinaturas{assinaturas.pdf}

\resumo{Este trabalho é um modelo em {\latex} utilizando a classe \iftex.
Tal classe foi desenvolvida com base no manual de normalização de trabalhos acadêmicos do IFMG e nas normas relacionadas da Associação Brasileira de Normas Técnicas.
Este modelo apresenta uma estrutura básica com exemplos de elementos pré e pós-textuais.
Maiores informações sobre como utilizar a classe podem ser encontradas no manual da classe \iftex.}
\palavraschave{\iftex. Modelo. IFMG. \latex.}

\abstract{This work is a template in {\LaTeX} using the \iftex class.
This class was developed based on the academic work standardization manual of IFMG and the related norms of the Brazilian Association of Technical Standards.
This template presents a basic structure with examples of pre and post-textual elements.
Further information on how to use the class can be found in the \iftex class manual.}
\keywords{\iftex. Template. IFMG. \latex.}

\listasiglas{%
 \begin{itemize}[]
  \item[ABNT] -- Associação Brasileira de Normas Técnicas
  \item[IFMG] -- Instituto Federal de Educação, Ciência e Tecnologia de Minas Gerais
  \item[TCC] -- Trabalho de conclusão de curso
 \end{itemize}
}

\begin{document}

\maketitle

\chapter{INTRODUÇÃO}

A introdução desempenha um papel fundamental na preparação do leitor para o conteúdo que será abordado.
Ela começa contextualizando o tema, fornecendo informações relevantes sobre o assunto, sua importância e seu contexto mais amplo na área de estudo.
Além disso, a introdução deve fornecer justificativas convincentes para a realização da pesquisa, identificando lacunas no conhecimento existente, relevância prática ou teórica do tema e importância potencial dos resultados.

Destaca-se que as contribuições esperadas do trabalho para a área de estudo, que podem incluir avanços teóricos e práticos, implicações políticas ou sociais, entre outros.
Por fim, a introdução é geralmente concluída com um parágrafo que resume brevemente o objetivo geral do trabalho, reiterando os objetivos estabelecidos anteriormente.
É essencial que essa seção seja redigida com clareza e coesão para capturar a atenção do leitor e estabelecer uma base sólida para o restante do trabalho.

Por fim, é importante observar o regulamento e as normas de formatação e de elaboração de trabalhos de conclusão de curso do IFMG \cite{ifmg:2020:manual,ifmg:2021:tcc}.
Além disso, é interessante consultar o manual da classe \iftex para conhecer mais sobre as configurações e exemplos de uso \cite{ribeiro:2024:iftex}.

\chapter{DESENVOLVIMENTO}

A seção de desenvolvimento apresenta e discute os resultados do trabalho.
Inicialmente, os resultados são apresentados de forma objetiva, seguidos por uma discussão que os relaciona aos objetivos e à revisão de literatura.
A interpretação dos resultados à luz das teorias é essencial, assim como a comparação com estudos anteriores.
Finalmente, é importante reconhecer as limitações do estudo e sugerir direções futuras.
Essa seção contribui para a compreensão do tema e o avanço do conhecimento na área.

\chapter{CONCLUSÃO}

A conclusão resume os principais pontos discutidos e apresenta as conclusões alcançadas a partir do trabalho.
Ela destaca as descobertas mais significativas, sua relação com a literatura existente e suas implicações práticas ou teóricas.
Além disso, a conclusão reafirma os objetivos do trabalho e sugere áreas para futuras investigações.
É importante evitar a introdução de novas informações e manter a conclusão concisa e alinhada com os objetivos e resultados do estudo.

\chapter*{REFERÊNCIAS}

\printbibliography

\end{document}
