\documentclass{iftex2024}

\addbibresource{referencias.bib}
\titulo{Modelo de Monografia}
\autor{Marcos Roberto Ribeiro}
\local{Bambuí -- MG}
\data{2024-04-18}

\campus{\textit{Campus} Bambuí}
\curso{Bacharelado}{Engenharia de Computação}
\titulacao{Bacharel}

\orientador[F]{Nome da Orientadora}
\coorientador{Nome da Coorientador}
\instituicaocoorientador{Instituição do Coorientador}
\membrobanca{Fulando de Tal}{Instituição do Fulano de Tal}
\membrobanca{Ciclano de Tal}{Instituição do Ciclano de Tal}
\fichacatalografica{ficha.pdf}
\assinaturas{assinaturas.pdf}

\dedicatoria{Dedico este trabalho à minha esposa e filhos, incentivadores e fontes inesgotáveis de apoio, amor e compreensão.}

\agradecimentos{Agradeço a toda à minha família, esposa, filhos, pais e minha irmã, por acreditarem em mim e pelo incentivo constante na realização deste trabalho.

Agradeço à minha orientadora, ao meu coorientador e a todos que contribuíram de alguma forma para a realização deste trabalho.}

\epigrafe{A tarefa mais importante de uma pessoa que vem ao mundo é criar algo.}{Paulo Freire}

\resumo{Este trabalho é um modelo em {\latex} utilizando a classe \iftex.
Tal classe foi desenvolvida com base no manual de normalização de trabalhos acadêmicos do IFMG e nas normas relacionadas da Associação Brasileira de Normas Técnicas.
Este modelo apresenta uma estrutura básica com exemplos de elementos pré e pós-textuais.
Maiores informações sobre como utilizar a classe podem ser encontradas no manual da classe \iftex.}
\palavraschave{\iftex. Modelo. IFMG. \latex.}

\abstract{This work is a template in {\LaTeX} using the \iftex class.
This class was developed based on the academic work standardization manual of IFMG and the related norms of the Brazilian Association of Technical Standards.
This template presents a basic structure with examples of pre and post-textual elements.
Further information on how to use the class can be found in the \iftex class manual.}
\keywords{\iftex. Template. IFMG. \latex.}

\listafiguras
\listaquadros
\listatabelas

\listasiglas{%
 \begin{itemize}[]
  \item[ABNT] -- Associação Brasileira de Normas Técnicas
  \item[IFMG] -- Instituto Federal de Educação, Ciência e Tecnologia de Minas Gerais
  \item[TCC] -- Trabalho de conclusão de curso
 \end{itemize}
}

\listasimbolos{%
 \begin{itemize}[]
   \item[$\mathbb{X}$] -- Variável X
   \item[$\mathsf{I\!R}$] -- Conjunto dos números reais
 \end{itemize}
}

\begin{document}

\maketitle

\chapter{INTRODUÇÃO}

\index{INTRODUÇÃO!exemplo de}
A introdução desempenha um papel fundamental na preparação do leitor para o conteúdo que será abordado.
Ela começa contextualizando o tema, fornecendo informações relevantes sobre o assunto, sua importância e seu contexto mais amplo na área de estudo.
Além disso, a introdução deve fornecer justificativas convincentes para a realização da pesquisa, identificando lacunas no conhecimento existente, relevância prática ou teórica do tema e importância potencial dos resultados.

Destaca-se que as contribuições esperadas do trabalho para a área de estudo, que podem incluir avanços teóricos e práticos, implicações políticas ou sociais, entre outros.
Por fim, a introdução é geralmente concluída com um parágrafo que resume brevemente o objetivo geral do trabalho, reiterando os objetivos estabelecidos anteriormente.
É essencial que essa seção seja redigida com clareza e coesão para capturar a atenção do leitor e estabelecer uma base sólida para o restante do trabalho.

Por fim, é importante observar o regulamento e as normas de formatação e de elaboração de trabalhos de conclusão de curso do IFMG \cite{ifmg:2020:manual,ifmg:2021:tcc}.
Além disso, é interessante consultar o manual da classe \iftex para conhecer mais sobre as configurações e exemplos de uso \cite{ribeiro:2024:iftex}.

\section{Objetivos}

Os objetivos definem claramente o propósito e as metas do trabalho, devem ser específicos, mensuráveis, alcançáveis, relevantes e limitados no tempo.
Os objetivos podem ser divididos em objetivo geral e objetivos específicos.

\subsection{Objetivos geral}

O objetivo geral é a meta principal do trabalho, definindo o propósito geral do estudo.
O objetivo geral deste trabalho é apresentar um modelo de documento usando a classe \iftex.

\subsection{Objetivos específicos}

Os objetivos específicos são metas detalhadas que precisam ser alcançadas para atingir o objetivo geral\footnote{Recomenda-se que não seja estabelecida uma quantidade muito grande de objetivos específicos.}.
Eles direcionam as ações do trabalho e fornecem uma estrutura clara para o trabalho.
Como exemplo podemos estabelecer os seguintes objetivos específicos:
\begin{enumerate}
 \item Apresentar exemplos de elementos pré-textuais;
 \item Mostrar uma estrutura básica de documento;
 \item Exemplificar o uso de elementos pós-textuais.
\end{enumerate}

\chapter{FUNDAMENTOS TEÓRICOS}

A seção de fundamentos teóricos fornece uma base teórica sólida para o estudo, contextualizando o trabalho dentro do corpo existente de conhecimento na área.
A seção de fundamentos teóricos fornece a base conceitual e contextual para o seu estudo.
É importante escrevê-la de forma clara, organizada e fundamentada em pesquisas anteriores, destacando a relevância e originalidade do seu trabalho.

\section{Identifique os principais temas e organize as informações}

Comece identificando os principais temas e conceitos relacionados ao seu tópico de pesquisa.
Isso pode envolver a leitura de artigos acadêmicos, livros e outras fontes relevantes.

Organize os temas e conceitos identificados de forma lógica e coerente.
Pode ser útil agrupar conceitos semelhantes e discutir suas inter-relações.

\section{Descreva teorias e modelos relevantes}

Descreva as teorias e modelos relevantes que fundamentam o seu estudo.
Explique como essas teorias se relacionam com o seu tema de pesquisa e como elas influenciam a sua abordagem metodológica.

\section{Apresente estudos anteriores e destaque as lacunas}

Revise estudos anteriores que são relevantes para o seu trabalho.
Discuta as descobertas desses estudos e como elas contribuem para o entendimento do seu tema de pesquisa.

Identifique lacunas no conhecimento existente e justifique como o seu estudo pretende preencher essas lacunas.
Isso demonstra a originalidade e importância do seu trabalho.

\section{Mantenha-se atualizado}

Certifique-se de incluir pesquisas recentes e relevantes na sua revisão de literatura.
Isso ajuda a garantir que o seu trabalho esteja atualizado e informado sobre os desenvolvimentos mais recentes na área.

\section{Cite corretamente as fontes}

Ao escrever a seção de fundamentos teóricos, lembre-se de citar corretamente todas as fontes utilizadas.
Isso inclui citar as obras de outros autores e fornecer referências bibliográficas completas.

\chapter{METODOLOGIA}

A metodologia descreve os métodos e procedimentos utilizados na pesquisa.
Ela inclui detalhes sobre o design do estudo, a coleta e análise de dados, além da justificativa das escolhas metodológicas.
É essencial para garantir a validade e confiabilidade dos resultados.
A metodologia deve ser clara e detalhada o suficiente para que outros pesquisadores possam replicar o estudo.

De acordo com \citet{ifmg:2021:tcc}, \enquote{Todos os trabalhos devem informar, no capítulo referente à Metodologia, a sua classificação quanto à natureza, objetivo, procedimentos e abordagem}.
Uma possível classificação pode ser feita quanto à abordagem.
Nesse caso, uma exemplo de classificação é a pesquisa qualitativa:
\begin{quote}
A pesquisa qualitativa preocupa-se com aspectos da realidade que não podem ser quantificados, centrando-se na compreensão e explicação da dinâmica das relações sociais.
Não se preocupa com representatividade numérica em si, mas com o aprofundamento da compreensão de um grupo social, de uma organização, etc.
\cite{ifmg:2021:tcc}.
\end{quote}

\chapter{DESENVOLVIMENTO}

A seção de desenvolvimento apresenta e discute os resultados do trabalho.
Inicialmente, os resultados são apresentados de forma objetiva, seguidos por uma discussão que os relaciona aos objetivos e à revisão de literatura.
A interpretação dos resultados à luz das teorias é essencial, assim como a comparação com estudos anteriores.
Finalmente, é importante reconhecer as limitações do estudo e sugerir direções futuras.
Essa seção contribui para a compreensão do tema e o avanço do conhecimento na área.

No desenvolvimento, podem ser utilizadas figuras, tabelas e quadros para ilustrar melhor a evolução do trabalho.
A Figura \ref{figura:logomarca_if} exibe a logomarca dos institutos federais.
Outros exemplos são a Tabela \ref{tabela:lista_produtos} e o Quadro \ref{quadro:editores_texto_livres}.

\begin{figure}[!htb] \centering
  \caption{Logomarca do IF} \label{figura:logomarca_if}
  \begin{varwidth}{\linewidth}
    \includegraphics[width=4cm]{figuras/if}
    \legend{\citefonte{ifmg:2020:manual}.}
  \end{varwidth}
\end{figure}

\begin{table}[!htb]
\caption{Lista de produtos} \label{tabela:lista_produtos}
\begin{tabularx}{\textwidth}{X|l|r|r|r} \hline
Produto      & Unidade & Preço (R\$) & Quantidade & Total (R\$) \\ \hline
Arroz        & Kg      & 2,00        & 550        & 1.100,00    \\
Óleo de Soja & L       & 2,50        & 500        & 750,00      \\
Açucar       & Kg      & 3,00        & 100        & 300,00      \\ \hline
\end{tabularx}
\legend{Elaborado pelo Autor, 2020.}
\end{table}

\begin{board}[!htb] \centering
\caption{Editores de Texto Livres} \label{quadro:editores_texto_livres}
\begin{varwidth}{\linewidth}
\begin{tabular}{|l|l|r|}        \hline
Editor     & Multiplataforma & Específico para Latex \\ \hline
Kwriter    & Sim             & Não                   \\
Texmaker   & Sim             & Sim                   \\
Kile       & Sim             & Sim                   \\
Geany      & Sim             & Não                   \\ \hline
\end{tabular}
\legend{Elaborado pelo Autor, 2020.}
\end{varwidth}
\end{board}

As equações devem ser apresentadas de forma centralizada e enumeradas quando necessário (ou seja, apenas se houver citação das equações no texto).

\begin{equation} \label{eq:vel_ondas}
 v=\frac{1}{\sqrt{\epsilon_0\mu_0}}
\end{equation}

A Equação \eqref{eq:vel_ondas} representa a velocidade das ondas eletromagnéticas, e a Equação \eqref{eq:fubini} refere-se ao famoso Teorema de Fubini, considerando $R=\left\{(x,y)~|~a\leq x\leq b,~c\leq y\leq d\right\}$ com $a,b,c,d\in\mathbb{R}$.

\begin{equation} \label{eq:fubini}
 V=\int\int\limits_R f(x,y)~dA=\int_a^b\int_c^d f(x,y)~dy~dx=\int_c^d\int_a^b f(x,y)~dx~dy
\end{equation}

\chapter{CONCLUSÃO}

\index{CONCLUSÃO!exemplo de}
\index{INTRODUÇÃO!conclusão amarrada com}
A conclusão resume os principais pontos discutidos e apresenta as conclusões alcançadas a partir do trabalho.
Ela destaca as descobertas mais significativas, sua relação com a literatura existente e suas implicações práticas ou teóricas.
Além disso, a conclusão reafirma os objetivos do trabalho e sugere áreas para futuras investigações.
É importante evitar a introdução de novas informações e manter a conclusão concisa e alinhada com os objetivos e resultados do estudo.

Após a conclusão são apresentados alguns exemplos de elementos pós-textuais.
Inclusive, elementos como apêndices e anexos devem ser referenciados.
Como exemplo, exitem o Apêndice \ref{ap:exemplo} e o Anexo \ref{an:exemplo}.

\chapter*{REFERÊNCIAS}

\printbibliography

\chapter*{GLOSSÁRIO}

\begin{itemize}[]
\item[LaTeX] -- Linguagem de marcação utilizada principalmente para a composição de documentos técnicos e científicos, fornecendo uma formatação consistente e de alta qualidade.
\item[Modelo / Template] -- Documento ou conjunto de elementos predefinidos que serve de estrutura base para a criação de outros documentos, permitindo uma formatação consistente e facilitando o trabalho de edição.
\end{itemize}

\appendix

\chapter{Exemplo de apêndice} \label{ap:exemplo}

Este é apenas um exemplo de apêndice.

\attachment

\chapter{Exemplo de anexo} \label{an:exemplo}

Este é apenas um exemplo de anexo.

\chapter*{ÍNDICE}

\printindex

\end{document}
